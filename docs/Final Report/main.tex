\documentclass[a4paper, 11pt]{article}

\usepackage[british]{babel}
\usepackage[autostyle]{csquotes}
\usepackage[colorlinks=true, urlcolor=blue, citecolor=blue, linkcolor=blue]{hyperref}
\usepackage{graphicx}
\usepackage{float}
\usepackage{geometry}
\usepackage[toc,page]{appendix}
\usepackage{caption}
\usepackage{subcaption}

\graphicspath{{images/}}
\geometry{margin=2.0cm}

\begin{document}

\title{Machine Learning to Study Patterns in Chess Games}
\author{Student Number: 690065435}
\date{Academic Year 2022/2023}

\maketitle

\begin{abstract}
{Abstract here}.

\begin{center}
\end{center}
\end{abstract}

\vspace*{\fill}
\begin{center}

\vspace{1em}
I certify that all material in this report which is not my own work has been identified.
\end{center}
\vspace{1em}

Signature: \hrulefill

\newpage

\section{Introduction}

\section{Project Specification}

\section{Design}

\subsection{Downloading the Data}
Collecting the data correctly was paramount to the success of our project, and it was important to use a large sample size to ensure that our insights represent the general population of chess games. We used the Lichess Open Database \cite{lichessOpenDatabase} of standard rated games for our data source -- they upload tens of millions of games every month in PGN format, and they are easily accessible to the public. We decided to focus on games in 2022, as this enables me to capture the latest trends in chess.

The time controls in Lichess are decided based on the estimated game duration, using the formula: $total \; time \; in \; seconds = (initial \; clock \; time) + 40 \times (clock \; increment)$. Games between 29 seconds and 179 seconds are Bullet; games between 179 seconds and 479 seconds are Blitz, games between 479 seconds and 1499 seconds are Rapid, and games over 1500 seconds are Classical. Furthermore, games are either Rated or Unrated -- the former results in rating points changing based on the game outcome, whereas the latter does not. The Lichess Open Database only includes Rated games. We will only be analysing Rated Bullet, Rated Blitz, and Rated Rapid games -- we will not be including Rated Classical games, as they have a significantly smaller sample size and the longer time controls result in a vastly different playstyle.

We started by downloading the PGN files for each month, which introduced difficulties with big data. Each month's file is around 30 GB to download and they are compressed, which means each month is around 210 GB when uncompressed, resulting in approximately 2.5 TB of data for the year. My laptop only had 1 TB of storage, and the sheer size of the data would make it unrealistic to process in a reasonable amount of time. Therefore, we collected a sample of 6 million chess games from each month which we would later filter down.

\subsection{Data Processing}
PGN (Portable Game Notation) files are a standard format for recording chess games -- each game has headers containing information like the white player, black player, their ratings, and the result of the game. In addition, it stores a list of all the moves in each game. While PGN files are well-structured, they are not easy to process. The pgn2data Python library \cite{pgn2dataGitHub} provides a simple way to convert PGN files into CSV files, but it exported both the moves and the metadata of each game in separate CSV files. We did not need the moves, so it was using up unnecessary space and taking a long time to process each game. It was also designed for generic PGN files so it did not include additional useful metadata provided by Lichess such as the opening of the game.

Thus, we wrote our own Python script to convert each PGN file into a single CSV file containing only the metadata of each game. We used the python-chess library to parse the PGN files and iterate over each game, skipping Rated Bullet games, Rated Classical games, games that involved a non-human player, and games that were abandoned. Our sample reduced from 60 million games to slightly over 40 million games after filtering it. To optimise performance, we split each month's PGN into six files each containing 1 million games with pgn-extract (a PGN manipulator written in C) \cite{pgnExtractGitHub}, and we used Python's multiprocessing library to process each month in parallel -- this reduced the processing time from 120 minutes to 20 minutes for each month.

Once we had a CSV for each month, we combined them into a single CSV file containing our final data set. We converted this CSV to a folder of Parquet files using Dask \cite{dask}, which is a library for parallel computing in Python and extends common interfaces like pandas \cite{pandas} to handle big data in a larger-than-memory environment. This significantly reduced the load time for the data, and it meant we could use the Dask library to perform parallel computations on the data.

For low-level, turn-based analysis of games, we created a .scout file using Scoutfish (a tool for querying chess databases written with C++) \cite{scoutfish} to enable fast queries. For example, we can query the database to find all games where the board state is a certain position using the FEN (Forsyth-Edwards Notation) of the position, which is the standard string used in chess to represent the board state -- this may be useful for finding occurrences of a specific variation of an opening. It provides the offset line number of each matching game in the PGN file, which we can use to find more information about the game. However, Scoutfish queries are not intuitive to write, so it can take a lot of effort to get insights using this tool. Therefore, we will primarily focus on high-level analysis of game metadata using Dask DataFrames.

\begin{figure}[H]
    \centering
    \caption{Diagram of the Data Pipeline}
    \label{fig:dataPipeline}
    \includegraphics[width=0.8\textwidth]{Data Pipeline.png}
\end{figure}

Figure \ref{fig:dataPipeline} shows an overview of our data pipeline. It shows how we used the Lichess Open Database to collect the data over 12 months. To analyse game metadata, we aggregated the data into a single CSV file, and converted it to a folder of Parquet files to query with Dask. Meanwhile, to perform more detailed analysis of game scenarios, we used this data to create a Scout index to enable fast queries.

\section{Data Exploration}

\subsection{Distribution of Player Ratings}
To start with, we counted the ratings of players in our data set. Rating systems are designed to predict the outcome of games so that players can be matched fairly. There are various rating systems used in chess -- the most common is the Elo rating system, which is used by various sports. Lichess uses the Glicko 2 rating system, which accounts for volatility amongst other factors to provide better prediction accuracy than the Glicko 1 and Elo rating systems \cite{chessRatingSystems, DeloitteFIDEChessRatingChallenge}. The Glicko 2 rating system starts at 1500, and it is artificially limited to a minimum of 600 in Lichess. We created bins of 200 rating point ranges starting from 600 -- this handles the representation of new players well, as those who have played few games will not have changed their rating significantly and likely get placed in the 1400-1600 bin. Note that the last four bins on the right of the bar chart do exist, but they are difficult to see -- they contain totals of 11,622 for 2800-3000, 382 for 3000-3200, 459 for 3200-3400, and 1 player for 3400-3600. In addition, the counts are based on each game played, so some players may be overrepresented in the data set -- a player who plays 100 games within the year will have 100 entries in the data set, whereas a player who plays 1 game will have 1 entry. However, this is not a significant issue because the data set is large enough to represent the distribution of player ratings well.

\begin{figure}[H]
    \centering
    \caption{Distribution of Player Ratings on Lichess in 2022}
    \label{fig:distributionOfPlayerRatings}
    \includegraphics[width=0.8\textwidth]{Distribution of Player Ratings.png}
\end{figure}

Figure \ref{fig:distributionOfPlayerRatings} shows that the distribution of player ratings tends towards a normal distribution. We could explain this based on the concept that a player's performance is similar to a random walk, which states that an object moves randomly with equal probability in different directions. In Lichess ratings, a player's rating can be seen as the result of a series of random walks, where the player's rating constantly changes as they win and lose games. Over time, their rating will tend to stabilise to represent their true skill level. Moreover, the central limit theorem states that the sum of a large number of independent and identically distributed random variables tends towards a normal distribution \cite{le1986central}. In this case, the ratings of individual players can be seen as independent and identically distributed random variables, with each game representing a single trial. Therefore, the central limit theorem predicts that the distribution of ratings will tend towards a normal dsitribution as the number of games increases. This is supported by how largest bin is the 1400-1600 bin, which contains the starting rating of 1500. Lichess' matchmaking system could contribute to this -- it matches players with similar skill, so games theoretically end up with a roughly equal number of wins and losses.

\subsection{Most Popular Openings by Category}
\begin{figure}[H]
    \centering
    \caption{Most Popular Openings by Category on Lichess in 2022}
    \label{fig:mostPopularOpeningsByCategory}
    \includegraphics[width=0.8\textwidth]{Most Popular Openings by Category.png}
\end{figure}

We also counted the most popular openings by category in figure \ref{fig:mostPopularOpeningsByCategory}, as this is represented by the ECO column provided in the Lichess PGN data. ECO (Encyclopedia of Chess Openings) is a classification of chess openings based on the first few moves of the game. Editors who mostly consist of chess grandmasters select critical opening lines and assign them a code \cite{matanovic1971classification}. The ECO system consists of five categories, A to E, each of which represent a category of opening. Within each category, there are further subcategories to group openings more explicitly. For example, a game with an ECO code of A41 means that the opening is a flank opening as it is in the A category, and the 41 means that the opening was the Queen's Pawn Game. The ECO system is not perfect as it does not include all openings (as demonstrated by the 32,799 games with the unknown category in figure \ref{fig:mostPopularOpeningsByCategory}), but it is useful to analyse opening trends.

Figure \ref{fig:mostPopularOpeningsByCategory} shows that the most popular category of openings are open games and semi-open games, whereas the Indian defences and closed games are the least popular. This makes sense for the general population of games -- the ideas behind these openings are simple and straightforward, as they focus on controlling the centre and developing pieces quickly. However, we see that the pattern is different for players rated 2000 and above -- semi-open games and flank openings are more popular than open games in this category (shown in figure \ref{fig:mostPopularOpeningsByCategoryRated2000Plus} from the appendices). At higher rated games, players often prefer to play more complex and dynamic openings that enable opportunities to create imbalances in the position and therefore gain an advantage at an earlier stage of the game.

\subsection{Most Popular Base Openings}
\begin{figure}[H]
    \centering
    \caption{Most Popular Openings on Lichess in 2022}
    \label{fig:mostPopularOpenings}
    \includegraphics[width=0.8\textwidth]{Most Popular Openings.png}
\end{figure}

The data set also includes a column for the specific opening detected in each game based on Lichess' opening detection. We grouped openings that were different variations of each other together to get a more accurate representation of the most popular openings -- figure \ref{fig:examplesOfChessOpenings} from the appendices contains examples of openings on the chess board. The top 15 most popular openings account for 61.79\% of all games in our data set. Figure \ref{fig:mostPopularOpenings} shows that the most popular openings are the Sicilian Defence and the Queen's Pawn Game by a significant margin. The Sicilian Defence is the most well-studied response to White moving the king's pawn forward by two, which could explain why it is the most popular opening. However, the Queen's Pawn Game is slightly misleading, as it describes any opening beginning with White moving the queen's pawn where they do not play the Queen's Gambit and therefore encompasses a large number of openings such as the London System which are not labelled separately.

\section{Conclusion}

\subsection{Limitations and Future Work}

\bibliography{main.bib}
\bibliographystyle{ieeetr.bst}

\newpage
\begin{appendices}

\section{Additional Plots}
\begin{figure}[H]
    \centering
    \caption{Most Popular Openings by Category on Lichess in 2022 (Rated 2000+)}
    \label{fig:mostPopularOpeningsByCategoryRated2000Plus}
    \includegraphics[width=\textwidth]{Most Popular Openings by Category (Rated 2000+).png}
\end{figure}

\section{Examples}
\begin{figure}[H]
    \centering
    \caption{Examples of Chess Openings}
    \label{fig:examplesOfChessOpenings}
    \begin{subfigure}{0.45\textwidth}
        \centering
        \caption{Sicilian Defence}
        \includegraphics[width=\textwidth]{Example of Sicilian Defence.png}
    \end{subfigure}
    \hfill
    \begin{subfigure}{0.45\textwidth}
        \centering
        \caption{Queen's Gambit}
        \includegraphics[width=\textwidth]{Example of Queen's Gambit.png}
    \end{subfigure}
\end{figure}

\end{appendices}

\end{document}